
% ----------------------- TODO ---------------------------
% Change per hand-in
\newcommand{\NUMBER}{1} % exercise set number
\newcommand{\EXERCISES}{5} % number of exercises

\newcommand{\COURSECODE}{AP3252}
\newcommand{\TITLE}{phase contrast transfer function}
\newcommand{\STUDENTA}{Jeroen Sangers}
\newcommand{\DEADLINE}{DEADLINE}
\newcommand{\COURSENAME}{Electron Microscopy: Characterisation of the nanoscale}
% ----------------------- TODO ---------------------------

\documentclass[a4paper]{scrartcl}

\usepackage[utf8]{inputenc}
\usepackage[british]{babel}
\usepackage{amsmath}
\usepackage{siunitx}
\usepackage{amssymb}
\usepackage{fancyhdr}
\usepackage{color}
\usepackage{graphicx}
\usepackage{lastpage}
\usepackage{listings}
\usepackage{tikz}
\usepackage{pdflscape}
\usepackage{subfigure}
\usepackage{float}
\usepackage{polynom}
\usepackage{hyperref}
\usepackage{tabularx}
\usepackage{forloop}
\usepackage{geometry}
\usepackage{listings}
\usepackage{fancybox}
\usepackage{tikz}
\usepackage{algpseudocode,algorithm,algorithmicx}
\usepackage{fontspec}


\setmainfont{Baskerville Light}[
	BoldFont	= Baskerville Bold ,
	ItalicFont	= Baskerville Light-Italic
]


% Algorithm command
\newcommand*\Let[2]{\State #1 $\gets$ #2}

% Matrix notation
\newcommand{\matr}[1]{\mathbf{#1}}

% Margins
\geometry{a4paper,left=3cm, right=3cm, top=3cm, bottom=3cm}

% Header and footer setup
\pagestyle {fancy}
%\fancyhead[L]{Tutor: \TUTOR}
\fancyhead[L]{\TITLE}
\fancyhead[C]{\STUDENTA}
\fancyhead[R]{\today}

\fancyfoot[L]{\COURSECODE}
\fancyfoot[C]{\COURSENAME}
\fancyfoot[R]{Page \thepage /\pageref*{LastPage}}

% Formatting of "title"
\def\header#1#2{
  \begin{center}
    {\Large Exercise set}\\
    {(Deadline #2)}
  \end{center}
}

\begin{document}
\subsubsection*{Exercise 1}
\begin{enumerate}
  \item For high $u$ the envelope function tends to zero, as such the phase correlation transfer function also tends to zero regardless of $\Delta f$.
  \item A zero crossing means that there is no phase contrast at these frequencies, such that there is no phase contrast image formed at these frequencies.
  \item A defocus of zero ($\Delta f = 0$) would have no zero crossing thus allowing for the capture of information at all frequencies. The magnitude of phase contrast is low but with a CCD camera this should be no issue.
  \item Increasing the tension of the electrons increases the with of the envelope function, allowing for a wider range of frequencies to be recorded. Changing $C_s$ to be larger increases the amount of zero crossings at high-$u$
  \item Something like $\Delta f = -86$ would be best since it almost has a zero-crossing at $u=3$ $[nm^{-1}]$ such that there is still information here but all other zero-crossings are pushed further to the high-$u$ regime. This way there is good phase contrast up to $u \approxeq 5$ $nm^{-1}$.
\end{enumerate}
\end{document}